\documentclass[11pt,a4paper,sans]{moderncv}
\usepackage[utf8]{inputenc}
\usepackage[russian]{babel}

\usepackage{enumitem}
\setlist[itemize]{nosep}

\moderncvstyle{classic}                        % style options are 'casual' (default), 'classic', 'oldstyle' and 'banking'
\moderncvcolor{blue}                          % color options 'blue' (default), 'orange', 'green', 'red', 'purple', 'grey' and 
\usepackage[scale=0.87]{geometry}
\setlength{\hintscolumnwidth}{3.5cm}           % if you want to change the width of the column with the dates

\firstname{Екатерина}
\familyname{Балакина}

\address{Санкт-Петербург}{Россия}
\phone{+79006522842}
\email{kato.balakina@gmail.com}
\renewcommand\refname{Selected publications}
\nopagenumbers{} % uncomment to suppress automatic page numbering for CVs longer than one page

\begin{document}
\makecvtitle

\section{Образование}
\cvitem{Бакалавр}{Санкт-Петербургский Государственный Университет}
\cvitem{}{Математико-механический факультет}
\cvitem{}{Кафедра системного программирования}
\cvitem{}{01.09.2015 - 31.08.2019}

\section{Опыт работы}
\cvitem{Лето 2019}{\href{https://yandex.com/company/}{\underline{Яндекс}} -- cтажер-разработчик интерфейсов}
\cvitem {} {Typescript, React, RxJS, Storybook}
\cvitem {} {Улучшила покрытие end-to-end тестами (TestCafe)}
\cvitem {} {Исправляла ошибки в коде, правила верстку в соответствии с замечаниями дизайнера. }
\cvitem {} {Добавила новую функциональность: множество мелких задач по запросам пользователей; фронтенд для работы с новыми возможностями бэкенда.}
\cvitem {}{}
\cvitem{Октябрь 2019} {\href{hwww.myget-it.com}{\underline{GET Information Technology}} -- веб-разработчик}
\cvitem {-- настоящее время} {React, Typescript, Redux, Redux\--Saga, Material UI, React Final Form}
\cvitem {} {Рефакторила код, искала и устраняла ошибки, реализовывала новые компоненты и исправляла старые в соответствии с новым дизайном, проводила код-ревью}
\cvitem {} {Исправляла модели, изменяла данные в БД (C\#, PostgreSQL)}
\cvitem {} {Реализовала модуль для продуктового тура на основе open source библиотеки reactour и машины состояний xState}
\cvitem {} {Добавила поддержку WebSocket на фронтенде}

\section{Другой опыт}
\cvitem{Лето 2014}{\textbf{Участник Летней Комьютерной Школы, параллель B'}}
\cvitem{2015}{\textbf{Победитель (1 место) регионального этапа Всероссийской Олимпиады Школьников по информатике}}
\cvitem{2016}{\textbf{Сборы по программированию Moscow Workshops ACM ICPC}}

\section{Выпускная квалификационная работа}
\cvitem{2019}{\textbf{Интеграция робототехнической ОС (ROS) с кибернетическим контроллером ТРИК}}
\cvitem{}{C/C++, Qt, ROS, ROS2, CoAP, MQTT-SN}

\section{Технические навыки}
\cvitem{}{Хорошее понимание основных алгоритмов и структур данных}
\cvitem{}{HTML, CSS, SCSS, JavaScript, Typescript, React, Redux, Redux\--Saga, Linux, Git}

\section{Языки}
\cvlanguage{Английский}{\textbf{B2/C1}}{}

\end{document}
